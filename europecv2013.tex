\documentclass[a4paper,12pt,nodocument,nologo,notitle]{europecv2013}
\usepackage[T1]{fontenc}
\usepackage[utf8x]{inputenc}
\usepackage{color}
\usepackage[dvipsnames]{xcolor}
\usepackage{tabu}
\usepackage{tikz}

% \usepackage{showframe}

\author{\small Giacomo Mazzamuto}
\title{\small Documentation of the \LaTeX\ class\\
		\Large \textbf{\texttt{europecv2013.cls}}\\
		\small \vspace{0.2cm} Version 0.1, 2014-03-01
}
\date{}

\newcommand{\bs}{\textbackslash}

\begin{document}

\maketitle

\begin{abstract}
This paper describes how to use europecv2013.cls, a \LaTeX\ document class for typesetting a curriculum vitae according to the Europass initiative of the European Commission, 2013 version. This is an unofficial implementation of that format.

\end{abstract}

\section{Introduction}

\section{Requirements and installation}
\section{Differences with europecv}
\section{Usage}
\subsection{Class options}
\subsection{Personal information}
\ecvname{Name Surname}
\ecvdateofbirth{1 January 1970}
\ecvtelephone[(+555) 555 555]{(+555) 555 555}
\ecvemail{myemailaddress@email.com}
\ecvaddress{rue Wiertz, B-1047 Brussels}

\begin{minipage}{20cm}
\begin{europecv}
\ecvpersonalinfo
\end{europecv}
\end{minipage}


\subsection{Sectioning and item commands}
Use these simple commands to add items and sections to your cv:\\
\begin{minipage}{\textwidth}
\begin{europecv}
 \ecvsection{ecvsection}
 \ecvtitle{}{ecvtitle}
 \ecvitem{}{ecvitem}
 \ecvblueitem{left text}{ecvblueitem}
\end{europecv}
\end{minipage}

\begin{description}
\item[\texttt{\bs ecvsection[vspace]\{title\}}] Starts a new section.
\item[\texttt{\bs ecvitem[vspace]\{left\}\{right\}}] Puts left text on the left part and right text on the right part.
\item[\texttt{\bs ecvtitle[vspace]\{left\}\{right\}}] As before, but..
\item[\texttt{\bs ecvblueitem[vspace]\{left\}\{right\}}]
\end{description}

Sectioning commands must be used inside the \texttt{europecv} environment, and text within the environment should be typeset only inside a sectioning command. All of them have an optional argument that specifies how much vertical space to leave \emph{before} that command. Note that this is different in the old \texttt{europecv} class, where the optional argument was used to specified the space left after an item.

\subsection{Defined colors}
The following colors are defined in the package:
{%

\newcommand{\colpic}[1]{\tikz \draw[fill=#1,ultra thick,#1] (0,0) rectangle (0.5,0.5);}
\newcommand{\colrow}[1]{\texttt{#1} & \colpic{#1} & \extractcolorspecs{#1}{\model}{\mycolor}%
    \convertcolorspec{\model}{\mycolor}{HTML}\tmp\texttt{\tmp}}
    
\renewcommand{\arraystretch}{1.5}
\newcommand{\mc}[2]{\multicolumn{#1}{#2}}
\begin{center}
\begin{tabu} {X[1,r,m] X[0.2,c,m] X[1,l,m]}
color name & sample & HTML code\\
\hline
\colrow{ecvrulecolor}\\
\colrow{ecvsectioncolor}\\
\colrow{ecvhighlightcolor}\\
\colrow{ecvtablebordercolor}\\
\colrow{ecvlanglinkcolor}\\
\colrow{ecvtextcolor}\\
\tabuphantomline
\end{tabu}
\end{center}
}%
You can use these colors wherever you want in your document like this
\begin{verbatim}
\textcolor{colorname}{ Some text }
\end{verbatim} 

\end{document}
