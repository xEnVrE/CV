\documentclass[a4paper,nodocument,nologo,notitle,narrow,12pt]{europecv2013}
\usepackage[T1]{fontenc}
\usepackage{lmodern}
\usepackage{tikz}
\usepackage{url}
\ifpdf
\hypersetup{
     colorlinks   = true,
     pdftex	  = true
}
\fi

% \usepackage{showframe}

\author{\small Giacomo Mazzamuto}
\title{\small Documentation of the \LaTeX\ class\\
		\Large \textbf{\texttt{europecv2013.cls}}\\
		\small \vspace{0.2cm} Version 0.1, 2014-03-01
}
\date{}

\newcommand{\bs}{\textbackslash}

\begin{document}

\maketitle

\begin{abstract}
This paper describes how to use europecv2013.cls, a \LaTeX\ document class for typesetting a curriculum vitae according to the Europass initiative of the European Commission, 2013 version. This is an unofficial implementation of that format.

\end{abstract}

\tableofcontents

\clearpage

\section{Introduction}
This class is an unofficial \LaTeX\ implementation of the standard model for curricula vit\ae\ (the \emph{Europass~CV\/}) as recommended by the European Commission. The Europass~CV replaces the European CV, launched in~2002. In 2013 a major revision of the Europass~CV came out, featuring a neater, more compact and somewhat fancier layout. This class is an implementation of the 2013 version of that layout and is based on the \textsf{europecv}\footnote{\url{http://ctan.org/pkg/europecv}} class which is an implementation of the previous layout.

The Europass~CV defines both the content and the layout of a curriculum vit\ae. The \textsf{europecv2013} class provides support for the latter, and for the former as far as personal information and spoken languages are concerned. If you want to know how the content of a Europass~CV is structured, refer to the documentation provided by the Europass website.\footnote{\url{https://europass.cedefop.europa.eu/en/documents/curriculum-vitae/templates-instructions}.}

This class tries to be as close as possible to the standard model without sacrificing flexibility. Although it is primarily intended for users of the European Union, the class can be used for any kind of curriculum vit\ae\ (possibly with the options \texttt{notitle} and \texttt{nologo}, see below), or even for other kinds of documents.

The main differences compared to the official model are the use of Helvetica (which should be in any standard \LaTeX\ distribution) instead of Arial, and the use of vector images instead of low-resolution bitmaps.

\section{Tutorial}
A minimal (empty) curriculum vit\ae\ can be obtained with the following code:
\begin{verbatim}
\documentclass[english,narrow,a4paper]{europecv2013}
\begin{document}
  \begin{europecv}
  \end{europecv}
\end{document}
\end{verbatim}

For a complete list of usable class options see section \ref{sec:classOptions}.

\subsection{Personal information}
\ecvname{Name Surname}
\ecvdateofbirth{1 January 1970}
\ecvtelephone[(+555) 555 555]{(+555) 555 555}
\ecvemail{myemailaddress@email.com}
\ecvaddress{rue Wiertz, B-1047 Brussels}

\ecvLeftColumnWidth{50mm}
\ecvColSep{10pt}
\begin{minipage}{20cm}
  \begin{europecv}
    \ecvpersonalinfo
  \end{europecv}
\end{minipage}


\subsection{Sectioning and item commands}
Sectioning commands should be used inside the \texttt{europecv} environment. All of them have an optional argument that specifies how much vertical space to leave \emph{before} that command. Note that this is the opposite of what happened in the old \texttt{europecv} class.

Use these simple commands to add items and sections to your cv:\\
\ecvLeftColumnWidth{50mm}
\begin{minipage}{\textwidth}
  \begin{europecv}
    \makeatletter
    \@ecvitemskiptrue
    \makeatother
    \ecvbigitem{Job applied for}{ecvbigitem}

    \ecvsection{ecvsection}
    \ecvtitle[0pt]{april 2012 -- april 2014}{ecvtitle}
    \ecvitem{}{ecvitem}
    \ecvitem{}{ecvitem}
    \ecvitem{}{ecvitem}
    \ecvblueitem{left text}{ecvblueitem}
  \end{europecv}
\end{minipage}

\begin{description}
\item[\texttt{\bs ecvsection[\textit{vspace}]\{\textit{title}\}}] Starts a new section with the given \textit{title}.
\item[\texttt{\bs ecvitem[\textit{vspace}]\{\textit{left}\}\{\textit{right}\}}] Puts \textit{left} text on the left part and \textit{right} text on the right part of the page. This is a standard cv item. Note that, in the 2013 version of the Europass cv, text on the left part is much less used than before. Still, it can be sometimes useful to put some text on the left.
\item[\texttt{\bs ecvtitle[\textit{vspace}]\{\textit{left}\}\{\textit{right}\}}] As above, but the text is typeset in blue and with a larger font on the right. This is used to highlight a block in your cv related to a same job or degree, with dates typically going on the left and job description or degree type on the right.
\item[\texttt{\bs ecvblueitem[\textit{vspace}]\{\textit{left}\}\{\textit{right}\}}] As standard cv item, but text on the left is typeset in blue.
\item[\texttt{\bs ecvbigitem[\textit{vspace}]\{\textit{left}\}\{\textit{right}\}}] As before, but typesets the left text in capital blue letters and 
\end{description}


\subsection{Language skills}
\begin{footnotesize}
  \hspace*{-6.4cm}
  \ecvLeftColumnWidth{75mm}
  \ecvColSep{10pt}
  \begin{minipage}{20cm}
    \begin{europecv}
    \ecvmothertongue{English}
    \ecvlanguageheader
    \ecvlanguage{French}{C1}{C2}{B2}{C1}{C2}
    \ecvlastlanguage{German}{A2}{A2}{A2}{A2}{A2}
    \ecvlanguagefooter
    \end{europecv}
  \end{minipage}
\end{footnotesize}


\section{Advanced usage}

\subsection{Class options}
\label{sec:classOptions}
\begin{description}
 \item[nologo]
 \item[notitle]
 \item[nodocument]
 \item[narrow]
 \item[bigfont]
 \item[english] 
 \item[italian]
 \item[debug] 
\end{description}

\subsection{Defined colors}
The following colors are defined in the package:
{%

\newcommand{\colpic}[1]{\tikz \draw[fill=#1,ultra thick,#1] (0,0) rectangle (0.5,0.5);}
\newcommand{\colrow}[1]{\texttt{#1} & \colpic{#1} & \extractcolorspecs{#1}{\model}{\mycolor}%
    \convertcolorspec{\model}{\mycolor}{HTML}\tmp\texttt{\tmp}}
    
\renewcommand{\arraystretch}{1.5}
\newcommand{\mc}[2]{\multicolumn{#1}{#2}}
\begin{center}
  \begin{tabu} {X[1,r,m] X[0.3,c,m] X[1,l,m]}
    color name & sample & HTML code\\
    \hline
    \colrow{ecvrulecolor}\\
    \colrow{ecvsectioncolor}\\
    \colrow{ecvhighlightcolor}\\
    \colrow{ecvtablebordercolor}\\
    \colrow{ecvlanglinkcolor}\\
    \colrow{ecvtextcolor}\\
    \tabuphantomline
  \end{tabu}
\end{center}
}%
You can use these colors wherever you want in your document like this
\begin{verbatim}
\textcolor{colorname}{ Some text }
\end{verbatim} 

\end{document}
