%% start of file `template.tex'.
%% Copyright 2006-2013 Xavier Danaux (xdanaux@gmail.com).
%
% This work may be distributed and/or modified under the
% conditions of the LaTeX Project Public License version 1.3c,
% available at http://www.latex-project.org/lppl/.


\documentclass[11pt,a4paper,sans]{moderncv}        % possible options include font size ('10pt', '11pt' and '12pt'), paper size ('a4paper', 'letterpaper', 'a5paper', 'legalpaper', 'executivepaper' and 'landscape') and font family ('sans' and 'roman')

% moderncv themes
\moderncvstyle{classic}                             % style options are 'casual' (default), 'classic', 'oldstyle' and 'banking'
\moderncvcolor{blue}                               % color options 'blue' (default), 'orange', 'green', 'red', 'purple', 'grey' and 'black'
%% \renewcommand{\familydefault}{\sfdefault}         % to set the default font; use '\sfdefault' for the default sans serif font, '\rmdefault' for the default roman one, or any tex font name
%\nopagenumbers{}                                  % uncomment to suppress automatic page numbering for CVs longer than one page

\AtBeginDocument{\settowidth{\hintscolumnwidth}{XXXXXXX}}


% character encoding
\usepackage[utf8]{inputenc}                       % if you are not using xelatex ou lualatex, replace by the encoding you are using
%\usepackage{CJKutf8}                              % if you need to use CJK to typeset your resume in Chinese, Japanese or Korean
%% \usepackage[T1]{fontenc}
%% \usepackage[defaultsans]{opensans}
%% \usepackage{helvet}
%% \setsansfont{Helvetica Light}


% adjust the page margins
\usepackage[scale=0.8]{geometry}
%\setlength{\hintscolumnwidth}{3cm}                % if you want to change the width of the column with the dates
%\setlength{\makecvtitlenamewidth}{10cm}           % for the 'classic' style, if you want to force the width allocated to your name and avoid line breaks. be careful though, the length is normally calculated to avoid any overlap with your personal info; use this at your own typographical risks...

\newcommand\cventryAlt[6]{%
  %% \cventry{#1}{#2}{#3}{#4}{#5}{\fontfamily{lmss}\selectfont#6}}
  \cventry{#1}{#2}{#3}{#4}{#5}{#6}}

% personal data
\name{Nicola A.}{Piga}
%% \title{Resumé title}                               % optional, remove / comment the line if not wanted
\address{Via S. Quirico, 19d}{16163, Genoa, Italy}{}% optional, remove / comment the line if not wanted; the "postcode city" and and "country" arguments can be omitted or provided empty
\phone[mobile]{+39~339~325~4342}                   % optional, remove / comment the line if not wanted
%% \phone[fixed]{+2~(345)~678~901}                    % optional, remove / comment the line if not wanted
%% \phone[fax]{+3~(456)~789~012}                      % optional, remove / comment the line if not wanted
\email{nicola.piga@iit.it}                               % optional, remove / comment the line if not wanted
%% \email{nicola.piga@iit.it}                         % optional, remove / comment the line if not wanted
%% \extrainfo{additional information}                 % optional, remove / comment the line if not wanted
\photo[75pt][0.0pt]{picture}                       % optional, remove / comment the line if not wanted; '64pt' is the height the picture must be resized to, 0.4pt is the thickness of the frame around it (put it to 0pt for no frame) and 'picture' is the name of the picture file
%% \quote{Some quote}                                 % optional, remove / comment the line if not wanted
\social[linkedin][www.linkedin.com/in/nicola-piga-73b91ab5/?locale=en_US]{nicola-piga-73b91ab5}
\social[github][www.github.com/xEnVrE]{xEnVrE}

% to show numerical labels in the bibliography (default is to show no labels); only useful if you make citations in your resume
%\makeatletter
%\renewcommand*{\bibliographyitemlabel}{\@biblabel{\arabic{enumiv}}}
%\makeatother
%\renewcommand*{\bibliographyitemlabel}{[\arabic{enumiv}]}% CONSIDER REPLACING THE ABOVE BY THIS

% bibliography with mutiple entries
%\usepackage{multibib}
%\newcites{book,misc}{{Books},{Others}}
%----------------------------------------------------------------------------------
%            content
%----------------------------------------------------------------------------------
\begin{document}
%\begin{CJK*}{UTF8}{gbsn}                          % to typeset your resume in Chinese using CJK
%-----       resume       ---------------------------------------------------------
\makecvtitle

%% \section{\textbf{Who I am}}
\vspace{-8mm}
I am a Post Doctoral researcher working in the Humanoid Sensing and Perception (HSP) research line within the Istituto Italiano di Tecnologia.
\par
My research deals with the development of \emph{6D object pose tracking} algorithms for humanoid robots that combine neural networks, Bayesian filtering, vision and tactile sensing.

\section{\textbf{Facts about me}}
\cventryAlt{}{}{}{}{}{{}% Detailed achievements:%
  \vspace{-8mm}
\begin{itemize}
  %% \setlength{\itemindent}{-.19in}
\item {\normalsize Researcher in humanoid robotics with a passion for state estimation applied to 6D object pose tracking}.
\end{itemize}
\vspace{2mm}
\begin{itemize}
\item {\normalsize Enthusiast Linux and \texttt{C++} user}.
\end{itemize}
\vspace{2mm}
\begin{itemize}
\item {\normalsize Hardware experience with several humanoid and industrial robotic platforms.}
\end{itemize}
\vspace{2mm}
\begin{itemize}
\item {\normalsize Main interests: Bayesian filtering, 6D object pose tracking, Machine learning-aided object pose tracking, Tactile sensing.}
\end{itemize}}

\section{\textbf{Current Position}}
\cventryAlt{May, 2022 - current}{Post Doctoral Researcher @ Humanoid Sensing and Perception (IIT)}{\newline Istituto Italiano di Tecnologia}{Genoa}{Italy.}{}

\section{\textbf{Publications}}
\cventryAlt{2022}{ROFT: Real-Time Optical Flow-Aided 6D Object Pose and Velocity Tracking
}{\textbf{Piga, N.}, Onyshchuk Y., Pasquale G., Pattacini, U. and Natale, L.}{}{IEEE Robotics and Automation Letters, vol. 7, no. 1, pp. 159-166, Jan. 2022}{}

\cventryAlt{2021}{A Differentiable Extended Kalman Filter for Object Tracking Under Sliding Regime}{\textbf{Piga, N.}, Pattacini, U. and Natale, L.}{}{Frontiers in Robotics and AI, Humanoid Robotics, Vol. 8, Pg. 251, 2021.}{}

\cventryAlt{2021}{Active Perception for Ambiguous Objects Classification}{Safronov, E., \textbf{Piga, N.}, Colledanchise, M. and Natale, L.}{}{Accepted at 2021 IEEE/RSJ International Conference on Intelligent Robots and Systems (IROS 2021).}{}

\cventryAlt{2021}{MaskUKF: An Instance Segmentation Aided Unscented Kalman Filter for 6D Object Pose and Velocity Tracking}{\textbf{Piga, N.}, Bottarel, F., Fantacci, C., Vezzani, G., Pattacini, U. and Natale, L.}{}{Frontiers in Robotics and AI, Humanoid Robotics, Vol. 8, Pg. 38, 2021.}{}

\cventryAlt{2019}{ Magnetic 3-axis Soft and Sensitive Fingertip Sensors Integration for the iCub Humanoid Robot}{Holgado, A. C., \textbf{Piga, N.}, Pradhono Tomo, T., Vezzani, G., Schmitz, A., Natale, L. and Sugano, S.}{}{Proc. IEEE-RAS International Conference on Humanoid Robotics, Toronto, Canada, 2019.}{}

\newpage
\section{\textbf{Skills}}
\cventryAlt{Software}{Programming languages}{}{\texttt{C++}, Python.}{}{}
\cventryAlt{}{Libraries}{}{Eigen, OpenCV, VTK, YARP, ROS.}{}{}
\cventryAlt{}{Simulators}{}{Gazebo.}{}{}
\cventryAlt{}{OS}{}{Linux, Windows.}{}{}
\cventryAlt{}{Build systems}{}{CMake.}{}{}
\cventryAlt{}{Version control systems}{}{Git.}{}{}
\vspace{3mm}
\cventryAlt{Hardware}{Robots}{}{iCub humanoid robot, Franka Emika Panda, RGB-D sensors, Tactile sensors (capacitive, magnetic-based, vision-based).}{}{}
\vspace{3mm}
\cventryAlt{Soft skills}{Work-related}{}{Problem solving, Teamwork, Leadership, Motivation.}{}{}
\cventryAlt{}{Languages}{}{English (independent user), Italian (mother-tongue).}{}{}

\section{\textbf{Past Experience}}
\cventryAlt{November, 2021 - April 2022}{Ph.D. Candidate @ Humanoid Sensing and Perception (IIT)}{}{\newline Istituto Italiano di Tecnologia}{Genoa, Italy.}{}{}
\vspace{-5mm}
\subsection{{\normalsize Ph.D. Thesis ``Hybrid Architectures for Object Pose and Velocity Tracking at the Intersection of Kalman Filtering and Machine Learning'' (\href{https://doi.org/10.15167/piga-nicola-agostino_phd2022-04-13}{Link})}}
\vspace{-5mm}
%% \cventryAlt{}{}{}{}{}{}
\vspace{10mm}
\cventryAlt{November, 2018 - November, 2021}{Ph.D. Student @ Humanoid Sensing and
  Perception (IIT)}{\newline Istituto Italiano di Tecnologia}{Genoa}{Italy.}{}{}
\vspace{-13mm}
\cventryAlt{}{}{}{}{}{I carried out my Ph.D. in Advanced and
  Humanoid Robotics at the Humanoid Sensing and Perception research line at
  Istituto Italiano di Tecnologia in Genoa. During the Ph.D. project, I developed
  hybrid architectures for 6D object pose tracking by fusing Deep Learning
  with Kalman filtering techniques. Part of my work was dedicated to the development
  of these algorithms in \texttt{C++} and their testing on the iCub humanoid platform
  using vision and tactile sensing.}
\vspace{5mm}
\cventryAlt{December 2017 - September 2018}{Research Fellow @ Humanoid Sensing and Perception (IIT)}{\newline Object localization using vision and touch:
  experiments on the iCub humanoid robot}{\newline Istituto Italiano di Tecnologia \& Università di Pisa}{Genoa, Italy.}{}{}
\vspace{-10mm}
\cventryAlt{}{}{}{}{}{During my M. Sc. thesis in collaboration with Istituto Italiano di Tecnologia (IIT), I designed a Bayesian object localization algorithm for the robot iCub exploiting visual and tactile measurements.}
%% \vspace{3mm}
%% \cventryAlt{July 2017}{\emph{Easy Peasy Robotics Coding Workshop} at Campus Party Italia}{Milan}{Italy.}{}{Two-days crash course about humanoid robot programming offered by the Istituto Italiano di Tecnologia. The course was organized as a set of lectures on robot control, robot vision and software architectures for robot programming followed by hands-on sessions using the simulator as well as a real iCub head.}
%% \vspace{3mm}
%% \cventryAlt{September, 2010 - November, 2010}{Internship at \href{http://www.connect.ie}{Connect.ie}}{}{Ireland.}{}{Development of websites for european projects.}


\section{\textbf{Education}}
%% \cventryAlt{July 2020}{Deep Learning: A Hands-on Introduction (University of Genoa)}{}{}{Genoa, Italy.}{}{}
%% \vspace{-5mm}
%% \cventryAlt{}{}{}{}{}{Deep Learning is a branch of Machine Learning that has recently achieved astonishing
%%   results in a number of different domains. This course provided a hands-on introduction to Deep Learning,
%%   starting from its foundations and discussing the various types of deep architectures and tools currently available.
%%   The theoretical classes were accompanied by work in lab giving the possibility of practicing deep learning
%%   with examples from real-world applications, with particular focus on visual data.
%% }
%% \cventryAlt{June 2020}{Regularization Methods for Machine Learning (RegML) (University of Genoa)}{}{}{Genoa, Italy.}{}{}
%% \vspace{-5mm}
%% \cventryAlt{}{}{}{}{}{Understanding how intelligence works and how it can be emulated by machines is an age old dream
%%   and arguably one of the biggest challenges in modern science. Learning, with its principles and computational implementations,
%%   is at the very core of this endeavor. Recently, for the first time, we have been able to develop artificial intelligence systems
%%   able to solve complex tasks considered out of reach for decades.
%%   \newline
%%   Starting from classical notions of smoothness, shrinkage and margin, the course covered state of the art techniques
%%   based on the concepts of geometry (aka manifold learning), sparsity and a variety of algorithms for supervised learning,
%%   feature selection, structured prediction, multitask learning and model selection. Classes on theoretical and algorithmic
%%   aspects were complemented by practical lab sessions.
%% }
%% \cventryAlt{September 2019}{Advancement of Artificial Intelligence WORKSHOP at Istituto Italiano di Tecnologia}{}{}{Genoa, Italy.}{}{}
%% \vspace{-5mm}
%% \cventryAlt{}{}{}{}{}{Workshop on recent advancement of AI in collaboration with Riken AIP.
%%   \newline
%% During the workshop I presented a poster entitled ``Instance Segmentation aided 6D object pose tracking using an Unscented Kalman Filter''.}
%% \cventryAlt{July 2019}{International Computer Vision Summer School 2019 (ICVSS) (University of Catania)}{}{}{Sicilia, Italy.}{}{}
%% \vspace{-5mm}
%% \cventryAlt{}{}{}{}{}{The thirteenth edition of the International Computer Vision Summer School aims to provide both an objective and clear overview and an in-depth analysis of the state-of-the-art research in Computer Vision. The last decade has seen a revolution in the theory and application of computer vision and machine learning. In this edition, the focus is on the emerging Computer Vision topics after Deep Learning revolution.
%%   \newline
%%   The courses were delivered by world renowned experts in the field, from both academia and industry, and covered both theoretical and practical aspects of real Computer Vision problems as well as examples of their successful commercialization.}
%% \vspace{3mm}
%% \cventryAlt{June 2019}{Machine Learning Crash Course 2019 (MLCC) (University of Genoa)}{}{}{Genoa, Italy.}{}{}
%% \vspace{-5mm}
%% \cventryAlt{}{}{}{}{}{Machine Learning is a key to develop intelligent systems and analyze data in science and engineering. Machine Learning engines enable
%%   intelligent technologies such as Siri, Kinect or Google self-driving car, to name a few. At the same time, Machine Learning methods help to decipher the
%%   information in our DNA and make sense of the flood of information gathered on the web, forming the basis of a new “Science of Data”.
%%   \newline
%%   This course provided an introduction to the fundamental methods at the core of modern Machine Learning covering theoretical foundations as well as essential
%%   algorithms. Classes on theoretical and algorithmic aspects were complemented by practical lab sessions.}
%% \vspace{3mm}
\cventryAlt{December, 2014 - September, 2018}{M. Sc. with full honours in Robotics and Automation Engineering}{\newline Università di Pisa}{Pisa, Italy.}{}{}
\vspace{-10mm}
\subsection{{\normalsize M. Sc. Thesis ``Object localization using vision and touch:
  experiments on the iCub humanoid robot''}}
%% \vspace{-5mm}
%% \cventryAlt{}{}{}{}{}{I developed a Bayesian filtering algorithm for object localization that uses visual and tactile information in the form of Cartesian points belonging to the surface of the object. To this end, I extended the state-of-the-art Memory Unscented Particle Filter algorithm for tactile localization of a stationary object in order to localize an object using visual measurements, in the form of point clouds, and track its pose using tactile measurements while the object is manipulated by an external end-effector. The algorithm was tested in simulation using the Gazebo simulator and on the iCub humanoid robot using its stereo vision and tactile sensing systems.}

%% \vspace{3mm}
%% \subsection{{\normalsize M. Sc. Projects:}}
%% %% \vspace{-5mm}
%% \cventryAlt{}{}{}{}{}{{}% Detailed achievements:%
%% \vspace{-8mm}
%% \begin{itemize}%
%% \item ``Design and implementation of an Auto-Ranging mechanism for the DecaWave EVB1000 indoor localization system''.
%%   \vspace{2mm}
%% \item ``Design and implementation of a hybrid position/force controller for the KUKA LWR4\texttt{+} manipulator equipped with a Pisa/IIT SoftHand in order to grasp thin objects exploiting safe hand-environment interactions''.
%% \item[] Video: \href{https://youtu.be/0tVq7SOc8s8}{https://youtu.be/0tVq7SOc8s8}.
%% \item[] \texttt{C++} implementation: \href{https://git.io/vdVYE}{https://git.io/vdVYE}.
%%   \vspace{2mm}
%% \item ``Robust Control of a Double Mass Spring Damper system''.
%%   \vspace{2mm}
%% \item ``Real time simulation of several flying 2D quadrotors each controlled along a user-defined trajectory using a LQG control system''.
%%   \vspace{2mm}
%% \item ``Simultaneous cooperative object localization and transport with several iRobot Create 2 robots in the Gazebo simulator''.
%% \item[] Python implementation: \href{https://git.io/v9KMv}{https://git.io/v9KMv}
%%   \vspace{2mm}
%% \item ``Measurement, transmission and representation of the attitude of a 3DoF mechanical system''.
%% \end{itemize}}

\vspace{5mm}
\cventryAlt{September, 2011 - December, 2014}{B. Sc. with full honours in Computer Engineering}{\newline Università di Pisa}{Pisa, Italy.}{}{}
\vspace{-10mm}
\subsection{{\normalsize B. Sc. Thesis ``Analysis of the reconstruction error in environmental monitoring via Compressive Sensing''}}
%% \vspace{-5mm}
%% \cventryAlt{}{}{}{}{}{I implemented a Compressive Sensing algorithm for the collection of environmental data and I carried out an analysis in order to compare the reconstruction error obtained with standard Nyquist based sampling techniques. The algorithm was tested with real measurements acquired using an Autonomous Weather Station.}

% Publications from a BibTeX file without multibib
%  for numerical labels: \renewcommand{\bibliographyitemlabel}{\@biblabel{\arabic{enumiv}}}% CONSIDER MERGING WITH PREAMBLE PART
%  to redefine the heading string ("Publications"): \renewcommand{\refname}{Articles}
%% \nocite{*}
%% \bibliographystyle{plain}
%% \bibliography{publications}                        % 'publications' is the name of a BibTeX file

% Publications from a BibTeX file using the multibib package
%\section{Publications}
%\nocitebook{book1,book2}
%\bibliographystylebook{plain}
%\bibliographybook{publications}                   % 'publications' is the name of a BibTeX file
%\nocitemisc{misc1,misc2,misc3}
%\bibliographystylemisc{plain}
%\bibliographymisc{publications}                   % 'publications' is the name of a BibTeX file

%\clearpage\end{CJK*}                              % if you are typesetting your resume in Chinese using CJK; the \clearpage is required for fancyhdr to work correctly with CJK, though it kills the page numbering by making \lastpage undefined
\end{document}


%% end of file `template.tex'.
