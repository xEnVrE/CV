%% start of file `template.tex'.
%% Copyright 2006-2013 Xavier Danaux (xdanaux@gmail.com).
%
% This work may be distributed and/or modified under the
% conditions of the LaTeX Project Public License version 1.3c,
% available at http://www.latex-project.org/lppl/.


\documentclass[11pt,a4paper,sans]{moderncv}        % possible options include font size ('10pt', '11pt' and '12pt'), paper size ('a4paper', 'letterpaper', 'a5paper', 'legalpaper', 'executivepaper' and 'landscape') and font family ('sans' and 'roman')

% moderncv themes
\moderncvstyle{classic}                             % style options are 'casual' (default), 'classic', 'oldstyle' and 'banking'
\moderncvcolor{blue}                               % color options 'blue' (default), 'orange', 'green', 'red', 'purple', 'grey' and 'black'
%% \renewcommand{\familydefault}{\sfdefault}         % to set the default font; use '\sfdefault' for the default sans serif font, '\rmdefault' for the default roman one, or any tex font name
%\nopagenumbers{}                                  % uncomment to suppress automatic page numbering for CVs longer than one page

\AtBeginDocument{\settowidth{\hintscolumnwidth}{XXXXXXX}}


% character encoding
\usepackage[utf8]{inputenc}                       % if you are not using xelatex ou lualatex, replace by the encoding you are using
%\usepackage{CJKutf8}                              % if you need to use CJK to typeset your resume in Chinese, Japanese or Korean
%% \usepackage[T1]{fontenc}
%% \usepackage[defaultsans]{opensans}
%% \usepackage{helvet}
%% \setsansfont{Helvetica Light}


% adjust the page margins
\usepackage[scale=0.8]{geometry}
%\setlength{\hintscolumnwidth}{3cm}                % if you want to change the width of the column with the dates
%\setlength{\makecvtitlenamewidth}{10cm}           % for the 'classic' style, if you want to force the width allocated to your name and avoid line breaks. be careful though, the length is normally calculated to avoid any overlap with your personal info; use this at your own typographical risks...

\newcommand\cventryAlt[6]{%
  %% \cventry{#1}{#2}{#3}{#4}{#5}{\fontfamily{lmss}\selectfont#6}}
  \cventry{#1}{#2}{#3}{#4}{#5}{#6}}

% personal data
\name{Nicola A.}{Piga}
%% \title{Resumé title}                               % optional, remove / comment the line if not wanted
\address{Via S. Quirico, 19d}{16163, Genova, Italy}{}% optional, remove / comment the line if not wanted; the "postcode city" and and "country" arguments can be omitted or provided empty
\phone[mobile]{+39~339~325~4342}                   % optional, remove / comment the line if not wanted
%% \phone[fixed]{+2~(345)~678~901}                    % optional, remove / comment the line if not wanted
%% \phone[fax]{+3~(456)~789~012}                      % optional, remove / comment the line if not wanted
\email{nicola.piga@iit.it}                               % optional, remove / comment the line if not wanted
%% \email{nicola.piga@iit.it}                         % optional, remove / comment the line if not wanted
%% \extrainfo{additional information}                 % optional, remove / comment the line if not wanted
\photo[75pt][0.0pt]{picture}                       % optional, remove / comment the line if not wanted; '64pt' is the height the picture must be resized to, 0.4pt is the thickness of the frame around it (put it to 0pt for no frame) and 'picture' is the name of the picture file
%% \quote{Some quote}                                 % optional, remove / comment the line if not wanted
\social[linkedin][www.linkedin.com/in/nicola-piga-73b91ab5/?locale=en_US]{nicola-piga-73b91ab5}
\social[github][www.github.com/xEnVrE]{xEnVrE}

% to show numerical labels in the bibliography (default is to show no labels); only useful if you make citations in your resume
%\makeatletter
%\renewcommand*{\bibliographyitemlabel}{\@biblabel{\arabic{enumiv}}}
%\makeatother
%\renewcommand*{\bibliographyitemlabel}{[\arabic{enumiv}]}% CONSIDER REPLACING THE ABOVE BY THIS

% bibliography with mutiple entries
%\usepackage{multibib}
%\newcites{book,misc}{{Books},{Others}}
%----------------------------------------------------------------------------------
%            content
%----------------------------------------------------------------------------------
\begin{document}
%\begin{CJK*}{UTF8}{gbsn}                          % to typeset your resume in Chinese using CJK
%-----       resume       ---------------------------------------------------------
\makecvtitle

\section{\textbf{Who I am}}
I am a first year \emph{PhD student} in \emph{humanoid robotics} with a background in \emph{automation} and \emph{robotics engineering}.
\par
My main research topic deals with \emph{multi-modal object perception and manipulation}. Currently, my focus is on the development of \emph{in-hand object tracking} algorithms for humanoid robots. 
\par
Although the object-tracking problem has been extensively addressed in the Computer Vision and Filtering literature, I think that still 
\par

\section{\textbf{6 facts about me}}
\cventryAlt{}{}{}{}{}{{}% Detailed achievements:%
  \vspace{-8mm}
\begin{itemize}
  %% \setlength{\itemindent}{-.19in}
\item {\normalsize Researcher in humanoid robotics with a passion for State estimation applied to visuo-tactile object tracking}.
\end{itemize}
\vspace{2mm}
\begin{itemize}
\item {\normalsize Involved in the development of a modern, cross-platform \texttt{C++} Bayesian estimation library}.
\end{itemize}
\vspace{2mm}
\begin{itemize}
\item {\normalsize Enthusiast Linux and \texttt{C++} user}.
\end{itemize}
\vspace{2mm}
\begin{itemize}
\item {\normalsize Good knowledge of YARP middleware and iCub libraries}.
\end{itemize}
\vspace{2mm}
\begin{itemize}
\item {\normalsize Hardware experience with the iCub humanoid robotic platform.}
\end{itemize}
\vspace{2mm}
\begin{itemize}
\item {\normalsize Main interests: Bayesian estimation, Object tracking, Object manipulation, Robotics simulation environments.}
\end{itemize}}

\section{\textbf{Current Position}}
\cventryAlt{November, 2018 - current}{PhD student @ Humanoid Sensing and Perception (Istituto Italiano di Tecnologia)}{Istituto Italiano di Tecnologia}{Genova}{Italy.}{I am a first year PhD student in Advanced and Humanoid Robotics at the Humanoid Sensing and Perception research line at Istituto Italiano di Tecnologia in Genova. My current research goal is to improve \emph{perception} capabilities of humandoid robots by combining visual and tactile measurements within object tracking algorithms. Part of my work is dedicated to the development of these algorithms in \texttt{C++} and their testing on the iCub humanoid platform. \newline{}% Detailed achievements:%
}

\section{\textbf{Skills}}
\cventryAlt{Software}{Programming languages}{}{\texttt{C++} (experienced), Python (prior experience), MATLAB (prior experience).}{}{}
\cventryAlt{}{Libraries}{}{Eigen (experienced), VCG (experienced), VTK (experienced), YARP (expert), OpenCV (basic knowledge).}{}{}
\cventryAlt{}{Simulators}{}{Gazebo (experienced).}{}{}
\cventryAlt{}{Tools}{}{Emacs (experienced), Meshlab (basic knowledge).}{}{}
\cventryAlt{}{OS}{}{Linux (experienced), Windows (basic knowledge).}{}{}
\cventryAlt{}{Build systems}{}{CMake (basic knowledge).}{}{}
\cventryAlt{}{Version control systems}{}{Git (experienced).}{}{}
\vspace{3mm}
\cventryAlt{Hardware}{Robots}{}{iCub humanoid platform (expert), industrial arms (prior experience).}{}{}
\vspace{3mm}
\cventryAlt{Soft skills}{Work-related}{}{Problem solving, Teamwork, Leadership, Motivation.}{}{}
\cventryAlt{}{Languages}{}{English (independent user), Italian (mother tounge).}{}{}

\section{\textbf{Achievements}}
\cventryAlt{2019}{In-hand object tracking for the iCub humanoid robot (\texttt{C++})}{}{}{}{I developed a Bayesian filtering algorithm for in-hand object-tracking that relies on Unscented Particle Filtering (UPF) speeded up using kd-trees. The algorithm processes partial point clouds of the object and contact points and uses a refined estimate of the robot hand in order to handle object occlusions. A \texttt{C++} implementation for the iCub humanoid robot is available at this website: \href{https://github.com/robotology-playground/visual-tactile-localization}{https://github.com/robotology-playground/visual-tactile-localization}.}

\section{\textbf{Past Experience}}
\cventryAlt{December 2017 - September 2018}{Research Fellow @ Humanoid Sensing and Perception (Istituto Italiano di Tecnologia)}{Object localization using vision and touch:
  experiments on the iCub humanoid robot}{Istituto Italiano di Tecnologia \& Università di Pisa}{Genova, Italy.}{}{}
\vspace{-10mm}
\cventryAlt{}{}{}{}{}{During my M. Sc. thesis in collaboration with Istituto Italiano di Tecnologia (IIT), I designed a Bayesian object localization algorithm for the robot iCub exploiting visual and tactile measurements.}
\vspace{3mm}
\cventryAlt{July 2017}{\emph{Easy Peasy Robotics Coding Workshop} at Campus Party Italia}{Milan}{Italy.}{}{Two-days crash course about humanoid robot programming offered by the Istituto Italiano di Tecnologia. The course was organized as a set of lectures on robot control, robot vision and software architectures for robot programming followed by hands-on sessions using the simulator as well as a real iCub head.}
\vspace{3mm}
\cventryAlt{September, 2010 - November, 2010}{Internship at \href{http://www.connect.ie}{Connect.ie}}{}{Ireland.}{}{Development of websites for european projects.}

\section{\textbf{Publications}}
\cventryAlt{2019}{In-hand object tracking using 3D visual and tactile measurements}{N. Piga, G. Vezzani, C. Fantacci, U Pattacini and L. Natale}{}{Submitted to IEEE/RSJ International Conference on Intelligent Robots and Systems (IROS), 2019.}{}

\section{\textbf{Education}}
\cventryAlt{December, 2014 - September, 2018}{M. Sc. with full honours in Robotics and Automation Engineering}{Università di Pisa}{Pisa, Italy.}{}{}
\vspace{-10mm}
\subsection{{\normalsize M. Sc. Thesis ``Object localization using vision and touch:
  experiments on the iCub humanoid robot''}}
\vspace{-5mm}
\cventryAlt{}{}{}{}{}{I developed a Bayesian filtering algorithm for object localization that uses visual and tactile information in the form of Cartesian points belonging to the surface of the object. To this end, I extended the state-of-the-art Memory Unscented Particle Filter algorithm for tactile localization of a stationary object in order to localize an object using visual measurements, in the form of point clouds, and track its pose using tactile measurements while the object is manipulated by an external end-effector. The algorithm was tested in simulation using the Gazebo simulator and on the iCub humanoid robot using its stereo vision and tactile sensing system.}

\vspace{3mm}
\subsection{{\normalsize M. Sc. Projects:}}
%% \vspace{-5mm}
\cventryAlt{}{}{}{}{}{{}% Detailed achievements:%
\vspace{-8mm}
\begin{itemize}%
\item ``Design and implementation of an Auto-Ranging mechanism for the DecaWave EVB1000 indoor localization system''.
  \vspace{2mm}
\item ``Design and implementation of a hybrid position/force controller for a the KUKA LWR4\texttt{+} manipulator equipped with a Pisa/IIT SoftHand in order to grasp thin objects exploiting safe hand-environment interactions''.
\item[] Video: \href{https://youtu.be/0tVq7SOc8s8}{https://youtu.be/0tVq7SOc8s8}.
\item[] \texttt{C++} implementation: \href{https://git.io/vdVYE}{https://git.io/vdVYE}.
  \vspace{2mm}
\item ``Robust Control of a Double Mass Spring Damper system''.
  \vspace{2mm}
\item ``Real time simulation of several flying 2D quadrotors each controlled along a user-defined trajectory using a LQG control system''.
  \vspace{2mm}  
\item ``Simultaneous cooperative object localization and transport with several iRobot Create 2 robots in the Gazebo simulator''.
\item[] Python implementation: \href{https://git.io/v9KMv}{https://git.io/v9KMv}
  \vspace{2mm}  
\item ``Measurement, transmission and representation of the attitude of a 3DoF mechanical system''.
\end{itemize}}

\vspace{5mm}
\cventryAlt{September, 2011 - December, 2014}{B. Sc. with full honours in Computer Engineering}{Università di Pisa}{Pisa, Italy.}{}{}
\vspace{-15mm}
\subsection{{\normalsize B. Sc. Thesis ``Analysis of the reconstruction error in environmental monitoring via Compressive Sensing''}}
\vspace{-5mm}
\cventryAlt{}{}{}{}{}{I implemented a Compressive Sensing algorithm for the collection of environmental data and I carried out an analysis in order to compare the reconstruction error obtained with standard Nyquist based sampling techniques. The algorithm was tested with real measurements acquired using an Autonomous Weather Station.}

% Publications from a BibTeX file without multibib
%  for numerical labels: \renewcommand{\bibliographyitemlabel}{\@biblabel{\arabic{enumiv}}}% CONSIDER MERGING WITH PREAMBLE PART
%  to redefine the heading string ("Publications"): \renewcommand{\refname}{Articles}
%% \nocite{*}
%% \bibliographystyle{plain}
%% \bibliography{publications}                        % 'publications' is the name of a BibTeX file

% Publications from a BibTeX file using the multibib package
%\section{Publications}
%\nocitebook{book1,book2}
%\bibliographystylebook{plain}
%\bibliographybook{publications}                   % 'publications' is the name of a BibTeX file
%\nocitemisc{misc1,misc2,misc3}
%\bibliographystylemisc{plain}
%\bibliographymisc{publications}                   % 'publications' is the name of a BibTeX file

%\clearpage\end{CJK*}                              % if you are typesetting your resume in Chinese using CJK; the \clearpage is required for fancyhdr to work correctly with CJK, though it kills the page numbering by making \lastpage undefined
\end{document}


%% end of file `template.tex'.
