% !TEX encoding = UTF-8
% !TEX program = pdflatex
% !TEX spellcheck = en_GB

\documentclass[english,a4paper]{europasscv}
\usepackage[english]{babel}

%%%%%%%%%%%%%%%%%%%%%%%%%%%%%%%%%%%%%%%%%%%%%%%%%%%%%%%%%%%%%%%%%%%%%%
%
% new commands
%
%%%%%%%%%%%%%%%%%%%%%%%%%%%%%%%%%%%%%%%%%%%%%%%%%%%%%%%%%%%%%%%%%%%%%%
%
\newcommand{\ecvbluetext}[1] {\textcolor{ecvsectioncolor}{#1}}
%
%%%%%%%%%%%%%%%%%%%%%%%%%%%%%%%%%%%%%%%%%%%%%%%%%%%%%%%%%%%%%%%%%%%%%%

%%%%%%%%%%%%%%%%%%%%%%%%%%%%%%%%%%%%%%%%%%%%%%%%%%%%%%%%%%%%%%%%%%%%%%
%
% personal info
%
%%%%%%%%%%%%%%%%%%%%%%%%%%%%%%%%%%%%%%%%%%%%%%%%%%%%%%%%%%%%%%%%%%%%%%
%
\ecvname{Nicola A. Piga}
\ecvaddress{Via Vi0co degli Adorno 2, Genova, Italy}
\ecvmobile{+39 339 325 4342}
\ecvemail{nicola.piga@iit.it}
\ecvgithubpage{www.github.com/xenvre}
\ecvlinkedinpage{www.linkedin.com/in/nicola-piga-73b91ab5}
\ecvdateofbirth{28 March 1992}
\ecvnationality{Italian}
\ecvgender{Male}
\ecvpicture[width=3.8cm]{picture.png}

%
%%%%%%%%%%%%%%%%%%%%%%%%%%%%%%%%%%%%%%%%%%%%%%%%%%%%%%%%%%%%%%%%%%%%%%

\begin{document}
\begin{europasscv}

  \ecvpersonalinfo

  %%%%%%%%%%%%%%%%%%%%%%%%%%%%%%%%%%%%%%%%%%%%%%%%%%%%%%%%%%%%%%%%%%%%%%
  %
  % education and training
  %
  %%%%%%%%%%%%%%%%%%%%%%%%%%%%%%%%%%%%%%%%%%%%%%%%%%%%%%%%%%%%%%%%%%%%%%
  %
  \ecvsection{Education and training}

  \ecvtitle{Dec 2014 -- current}{Master in Robotics and Automation Engineering}
  \ecvitem{}{University of Pisa \href{https://www.unipi.it/}{[ www.unipi.it ]}}
  \ecvblueitem[0pt]{Topics}{Mathematics (deterministic finite dimensional linear systems theory, probability),
    Control Theory (robust control and identification of linear systems, control of nonlinear systems, digital control),
    %Guidance, Navigation and Control, Control of Processes,
    Modelling (aerospace systems, underwater systems, vehicles),
    Robotics (robot kinematics, dynamics and control, distributed robotics),
    Computer Science (real time systems),
    Mechatronics,
    Fundamentals of power electronics.}

  \ecvtitle{Sept 2011 -- Dec 2014}{Bachelor of Science in Computer Engineering}
  \ecvitem{}{University of Pisa \href{https://www.unipi.it/}{[ www.unipi.it ]}}
  \ecvblueitem[0pt]{Graduation Mark}{110/110 cum laude}
  \ecvblueitem[0pt]{Thesis title}{Analysis of the reconstruction error in environmental monitoring via Compressive Sensing}
  %
  %%%%%%%%%%%%%%%%%%%%%%%%%%%%%%%%%%%%%%%%%%%%%%%%%%%%%%%%%%%%%%%%%%%%%%

  %%%%%%%%%%%%%%%%%%%%%%%%%%%%%%%%%%%%%%%%%%%%%%%%%%%%%%%%%%%%%%%%%%%%%%
  %
  % personal skills
  %
  %%%%%%%%%%%%%%%%%%%%%%%%%%%%%%%%%%%%%%%%%%%%%%%%%%%%%%%%%%%%%%%%%%%%%%
  %
  \ecvsection{Personal skills}

  \ecvblueitem{Programming skills}{Languages: C++, Python}
  \ecvitem{}{Tools: Emacs}
  %% \ecvitem{}{Libraries: Qt}
  \ecvitem{}{Version control systems: Git}

  \ecvblueitem{Robotics and Control skills}{Software: ROS, Gazebo}
  \ecvitem{}{Systems: Industrial arms}
  \ecvitem{}{Libraries: KDL}
  \ecvitem{}{Simulation: MATLAB, Simulink, Mathematica}

  \ecvmothertongue{Italian}
  \ecvlanguageheader
  \ecvlastlanguage{English}{B1}{B1}{B1}{B1}{B1}
  \ecvlanguagefooter

  %% \ecvblueitem{Driving licence}{Italian car driving license}
  %
  %%%%%%%%%%%%%%%%%%%%%%%%%%%%%%%%%%%%%%%%%%%%%%%%%%%%%%%%%%%%%%%%%%%%%%

  \pagebreak
  %%%%%%%%%%%%%%%%%%%%%%%%%%%%%%%%%%%%%%%%%%%%%%%%%%%%%%%%%%%%%%%%%%%%%%
  %
  % didactic projects 
  %
  %%%%%%%%%%%%%%%%%%%%%%%%%%%%%%%%%%%%%%%%%%%%%%%%%%%%%%%%%%%%%%%%%%%%%%
  %
  \ecvsection{Didactic projects}

  \ecvblueitem{Apr 2017 -- Oct 2017}{\ecvbluetext{\emph{Development of an Auto-Ranging feature for the DecaWave EVB1000 system } }}
  \ecvblueitem[1pt]{Course}{Guidance, Navigation and Control}
  \ecvitem{}{
    Design and implementation of an Auto-Ranging procedure for the DecaWave EVB1000 indoor localization system.
    \par
    The system was provided with a 3D Viewer that shows the results of the Auto-Ranging procedure and the position of one or more localized agents.
    \par
    Firmware programming was done in C using the DecaWave SDK.
    \par
    The 3D Viewer was written in Python.
    \par
    code: \href{https://git.io/vdV3n}{https://git.io/vdV3n} (3D Viewer only)
  }


  \ecvblueitem{Jan 2017 -- Mar 2017}{\ecvbluetext{\emph{Hybrid Position/Force control of a Kuka LWR4+ manipulator} }}
  \ecvblueitem[1pt]{Course}{Control of Robots}
  \ecvitem{}{
    Implementation of a hybrid position/force controller for a lightweight manipulator
    equipped with a Pisa/IIT SoftHand. The hybrid approach was used to exploit
    hand-environment interactions, which are required to perform the grasp of thin objects,
    in a safe way.
    \par
    The controller was developed using ROS (C++), simulated in Gazebo and tested
    on a real manipulator. The system was provided with an operator GUI developed in Qt.
    \par
    video: \href{https://youtu.be/0tVq7SOc8s8}{https://youtu.be/0tVq7SOc8s8}
    \par
    code: \href{https://git.io/vdVYE}{https://git.io/vdVYE}
  }

  \ecvblueitem{Set 2016 -- Nov 2016}{\ecvbluetext{\emph{Robust Control of a Double Mass Spring Damper system} [simulation]}}
  \ecvblueitem[1pt]{Course}{Identification and Control of Uncertain Systems}
  \ecvitem{}{
    Development of several control systems that regulate the position
    of the second mass in response to a step input in presence of load disturbance,
    measurement noise and parametric uncertainties.
    \par
    Several control techniques were used such as LQG/LTR, $\mathcal H_{\infty}$ and $\mu$  synthesis.
    \par
    Stability and performance robustness were tested using Matlab Robust Control Toolbox and
    simulation was performed using Simulink.
  }

  \ecvblueitem{May 2016 -- Jul 2016}{\ecvbluetext{\emph{Real Time simulation of flying 2D quadrotors in presence of noisy measurements}}}
  \ecvblueitem[1pt]{Course}{Real Time Systems}
  \ecvitem{}{
    Real time simulation of several flying 2D quadrotors each equipped
    with a LQG control system. The trajectory of each vehicle can be random or
    specified by the user.
    \par
    The software was written in C using the Pthreads library.
    The GUI was developed using the game programming library Allegro 4.
  }
  
  \ecvblueitem{Jan 2016 -- Mar 2016}{\ecvbluetext{\emph{Cooperative Transport with iRobot Create 2 robots} [simulation]}}
  \ecvblueitem[1pt]{Course}{Distributed Robotic Systems}
  \ecvitem{}{
    Development of a distributed system composed of three
    wheeled robots that move a box to a given position cooperatively in a
    space without constraints.
    \par
    Key concepts explored in the project are: point to point control of the robots, planning and obstacle
    avoidance, online distributed estimation of the position of the box.
    \par
    The system was developed using ROS (Python) with the planning library OMPL and simulated in Gazebo.
    \par
    code: \href{https://git.io/v9KMv}{https://git.io/v9KMv}
  }

  \ecvblueitem{Jan 2015 -- Apr 2015}{\ecvbluetext{\emph{Measurement, transmission and representation of the attitude of a 3DoF mechanical system}}}
  \ecvblueitem[1pt]{Course}{Mechatronics}
  \ecvitem{}{
    Implementation of the firmware for a WiFi enabled microcontroller that
    acquires the signals from three encoders, representing
    the attitude of a mechanical system, and send them via UDP.
    \par
    The system was provided with several tools: a web server, developed as a part
    of the firmware, for the configuration and a standalone
    GUI developed in Python to represent the attitude of the system.
    \par
    Firmware programming was done in C using the Broadcom WICED SDK.
  }
  %
  %%%%%%%%%%%%%%%%%%%%%%%%%%%%%%%%%%%%%%%%%%%%%%%%%%%%%%%%%%%%%%%%%%%%%%

  %%%%%%%%%%%%%%%%%%%%%%%%%%%%%%%%%%%%%%%%%%%%%%%%%%%%%%%%%%%%%%%%%%%%%%
  %
  % experiences 
  %
  %%%%%%%%%%%%%%%%%%%%%%%%%%%%%%%%%%%%%%%%%%%%%%%%%%%%%%%%%%%%%%%%%%%%%%
  %
  \ecvsection{Experiences}

  \ecvblueitem{Jul 2017}{\ecvbluetext{\emph{Easy Peasy Robotics Coding Workshop} at Campus Party Italia, Milan, Italy}}
  \ecvitem{}{Two-days crash course about humanoid robot programming offered by the Istituto Italiano di Tecnologia (IIT).
    The course was organized as a set of lectures on robot control, robot vision and software architectures for robot programming
    followed by hands-on sessions using the simulator as well as a real iCub head.}

  \ecvblueitem{Sep 2010 -- Nov 2010}{\ecvbluetext{\emph{Internship} at Connect.ie \href{http://www.connect.ie}{[ www.connect.ie ]}}}
  \ecvitem{}{Development of websites for european projects.}
  %
  %%%%%%%%%%%%%%%%%%%%%%%%%%%%%%%%%%%%%%%%%%%%%%%%%%%%%%%%%%%%%%%%%%%%%%

  %%%%%%%%%%%%%%%%%%%%%%%%%%%%%%%%%%%%%%%%%%%%%%%%%%%%%%%%%%%%%%%%%%%%%%
  %
  % personal interests 
  %
  %%%%%%%%%%%%%%%%%%%%%%%%%%%%%%%%%%%%%%%%%%%%%%%%%%%%%%%%%%%%%%%%%%%%%%
  %
  %% \ecvsection{Personal interests}
  %
  %%%%%%%%%%%%%%%%%%%%%%%%%%%%%%%%%%%%%%%%%%%%%%%%%%%%%%%%%%%%%%%%%%%%%%


\end{europasscv}

\end{document}
